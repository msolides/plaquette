\subsection{Endommagement}

\paragraph{Professeurs:}

\begin{itemize}
\item 
Hélène Dumontet, \url{http://www.dalembert.upmc.fr/home/dumontet},\\ E-mail : \href{mailto:helene.dumontet@sorbonne-universite.fr}{helene.dumontet@sorbonne-universite.fr}

\item
Jérémy Bleyer, \url{https://navier.enpc.fr/BLEYER-Jeremy},\\ E-mail : \href{mailto:jeremy.bleyer@enpc.fr}{jeremy.bleyer@enpc.fr}
\end{itemize}

\paragraph{Objectifs de l'Unité d'Enseignement}
Les objectifs de cet enseignement sont: 
\begin{itemize}
	\item de fournir les bases théoriques de la mécanique de l’endommagement des matériaux quasi-fragiles, en particulier concernant la formulation de lois de comportements macroscopiques couplant  élasticité et endommagement. 
 	\item d’étudier la problématique de l’initiation et de l’évolution de l’endommagement dans un cadre numérique afin d’aborder le caractère mal-posé des modèles d’endommagement locaux puis de proposer une ouverture vers plusieurs techniques de régularisation et les liens vers les modèles de rupture fragile.
\end{itemize}

\paragraph{Contenu de l’Unité d’Enseignement}
Après une brève introduction sur l’origine microscopique de l’endommagement comme processus faisant évoluer les propriétés macroscopiques des matériaux, les séances de ce cours seront dédiées à:

\begin{itemize}
 \item  la formulation de la loi de comportement élastique-endommageable dans le cadre des processus thermodynamiques irréversibles (matériaux standards généralisés)
 \item l’introduction de la notion de critère d’endommagement (surface seuil), de force thermodynamique associée (taux de restitution d’énergie) et de loi d’évolution de l’endommagement
 \item l’implémentation simple d’un modèle d’endommagement isotrope dans un code de calcul aux éléments finis (FEniCS)
 \item l’étude du caractère mal-posé des modèles d’endommagement locaux (dépendance au maillage) et une présentation de différentes techniques de régularisation (modèles non-locaux)
 \item une mise en oeuvre numérique de modèles à gradient d’endommagement (phase-field) pour la simulation de propagation de fissure dans les matériaux fragiles
\end{itemize}

\paragraph{Pré-requis:} Mécanique des milieux continus (comportement élastique), thermodynamique, calcul numérique (méthode des éléments finis)


\paragraph{Compétences développées dans l’unité}
Modélisation de l’endommagement, calcul de structures endommagées, propagation de fissure.
\paragraph{Références bibliographiques:}
{\footnotesize
\begin{enumerate}
\item Lemaitre, J., Chaboche, J. L., Benallal, A., \& Desmorat, R. (2009). Mécanique des matériaux solides-3eme  édition. Dunod.     
\item Pijaudier-Cabot G., Mazars J. (2001). Damage models for concrete. in Handbook of Materials Behavior. Vol. II, Lemaitre J. (ed.), Academic Press
\item Marigo, J. J., Maurini, C., \& Pham, K. (2016). An overview of the modelling of fracture by gradient damage models. Meccanica, 51(12), 3107-3128.
\end{enumerate}