
\subsection{Calcul numérique des solides et structures non linéaires}

\paragraph{Professeur(s):} 
\begin{itemize}
\item Denis Duhamel, \url{https://www.enpc.fr/denis-duhamel}
\item Corrado Maurini, \url: \url{http://www.lmm.jussieu.fr/~corrado} %\href{mailto:corrado.maurini@sorbonne-universite.fr}{corrado.maurini@sorbonne-universite.fr}
\end{itemize}

\paragraph{Objectifs de l'Unité d'Enseignement: }
L’enseignement a pour but de résoudre des problèmes de mécanique des solides non linéaires sur ordinateur en implémentant la méthode des éléments finis et les algorithmes de résolution. 
Les étudiants seront initiés à l’utilisation des outils modernes de calcul scientifique à haute performance (FEniCS, PETSc) des outils de visualisation (paraview) et de gestion de projet (git).

\paragraph{Contenu de l’Unité d’Enseignement:}

\begin{itemize}
\item Elasticité linéaire, formulation variationnelle, discrétisation:  Rappel sur la résolution d’un problème de mécanique par éléments finis. Prise en main de python et FEniCS, Résolution d’un problème d’élasticité par FEniCS
\item	Elasticité non linéaire:  Elasticité non-linéaire, linéarisation, flambement, stabilité. Résolution d’un problème d’élasticité non linéaire, flambement et post-flambement
\item	Dynamique non linéaire: Méthodes implicite et explicite. Résolution d’un problème de dynamique non linéaire
\item	Projet: travail sur un projet en binôme
\end{itemize}

\paragraph{Pré-requis:}
\begin{itemize}
\item Mécanique des milieux continus
\item Connaissance d’un langage de programmation (idéalment python),
\item Cours de base d’élements finis et méthodes numériques
\end{itemize}


\paragraph{Compétences développées dans l’unité}
\begin{itemize}
\item Capacité de développer un code numérique basé sur la méthode des éléments finis pour résoudre un problème d’élasticité linéaire et non-linéaire en statique ou dynamique en utilisant le language python et la librarie FEniCS
\item Elasticité nonlinéaire,  méthodes explicites et implicites pour la dynamique des structures, étude numérique de bifurcation et stabilité dans le cadre quasi-statique.
\item Introduction aux systèmes de gestion de révision (git), Introduction aux outils de visualisation et maillage, Utilisation des solveurs de systèmes d’équations linéaires et non-linéaires à grande dimension
\end{itemize}

\paragraph{Références bibliographique:}
{\footnotesize
\begin{enumerate}
	\item Belytschko T., Liu W. K. and Moran B., Non linear finite elements for continua and structures, 2000, Wiley. 
 \item Dhatt G., Touzot G. et Lefrancois E., Une présentation de la méthode des éléments finis, 2005, Hermes.
 \item  Holzapfel G.A., Nonlinear solid mechanics, 2000, Wiley.
 \item Scott R., Introduction to Automated Modeling with Fenics, 2018, Computational Modeling Initiative LLC.
 \item Langtange, P. and Logg A., Solving PDEs in Python, 2017, Springer.
 \item Davide Bigoni 
  Nonlinear Solid Mechanics
  Bifurcation Theory and Material Instability, 2012, 
   Cambridge University Press.
 \item Wriggers, P., Nonlinear finite element method, 2008, Springer.
 \item Bonnet M., Frangi A., Rey C., The finite element method in solid mechanics, 2014, McGraw Hill.
\end{enumerate}
}
