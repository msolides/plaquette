\subsection{Fatigue}

\paragraph{Professeur:} 
Michael PEIGNEY, \url{https://navier.enpc.fr/PEIGNEY-Michael},\newline E-mail: michael.peigney@enpc.fr


\paragraph{Objectifs de l'Unité d'Enseignement:}
La maîtrise par l’ingénieur des problèmes de conception de structures mécaniques exige de connaître et comprendre tous les modes possibles de défaillance. Ce cours cible les modes de défaillance par fatigue, susceptibles d’intervenir pour les structures soumises à des sollicitations variables en temps.
A l’issue de l’Unité d’Enseignant, les étudiants :
(i) connaîtront les concepts de base de la fatigue et de l’accumulation de l’endommagement dans les matériaux et les structures sous chargement cycliques et éventuellement aléatoires;
(ii)	connaîtront les principaux critères utilisés dans le milieu industriel; (iii)	seront capables d’en déduire la durée de vie d’une structure sous  chargement cyclique  
(iv)	sauront expliquer les causes des défaillances constatées en service.




\paragraph{Contenu de l’Unité d’Enseignement:}
\begin{itemize}
\item	\emph{Phénomènes physiques.}
Introduction de la problématique à partir d’un cas pratique: essieu-roue-rail. Mise en évidence du phénomène de fatigue. Courbes de Wöhler. Distinction entre fatigue à faible et à grand nombre de cycles. Mécanismes physiques de la fatigue (plasticité, fissuration, distinction entre amorçage et propagation).

\item	\emph{Chargements cycliques et théorèmes d’adaptation.}
Classification des comportements cycliques en élastoplasticité (adaptation, accommodation, rochet). Théorèmes de convergence aux temps longs. Théorèmes d’adaptation (statique et cinématique). Extensions hors de la plasticité parfaite. 

\item	\emph{Critère de fatigue.} Endurance illimitée
Critère de fatigue à grand nombre de cycles, en chargement uniaxial (parabole de Gerber, droite de Goodman) et multiaxial (notamment critères de Sine, Crossland, Dang Van). 

\item	\emph{Critère de fatigue.} Endurance limitée. Chargements aléatoires 
Lois d’endurance limitée. Durée de vie en fatigue à faible nombre de cycles (loi de Manson-Coffin, critère énergétique). Règles de cumul de Miner. Méthode Contrainte-Résistance. Comptage de type rainflow.


\end{itemize}

\paragraph{Pré-requis:} Mécanique de milieux continus. Des notions de plasticité sont souhaitables.


\paragraph{Compétences développées dans l’unité:}

Compréhension du comportement de structures élasto-plastiques sous chargement cycliques et des modes de défaillance associés, Dimensionnement de structures à la fatigue. 




\paragraph{Références bibliographiques:}
{\footnotesize
\begin{enumerate}

\item Constantinescu, A., K. Dang Van, and M. H. Maitournam. "A unified approach for high and low cycle fatigue based on shakedown concepts." Fatigue \& fracture of engineering materials \& structures 26.6 (2003): 561-568.

\item Bertolino, G., et al. "A multiscale approach of fatigue and shakedown for notched structures." Theoretical and Applied Fracture Mechanics 48.2 (2007): 140-151.

\item Peigney, Michael. "Shakedown theorems and asymptotic behaviour of solids in non-smooth mechanics." European Journal of Mechanics-A/Solids 29.5 (2010): 784-793.

\item Peigney, Michael. "Shakedown of elastic-perfectly plastic materials with temperature-dependent elastic 
moduli." Journal of the Mechanics and Physics of Solids 71 (2014): 112-131. 

\item Papadopoulos, Ioannis V., et al. "A comparative study of multiaxial high-cycle fatigue criteria for metals." International Journal of Fatigue 19.3 (1997): 219-235.

\item Papadopoulos, Ioannis V. (editor) Multiaxial fatigue limit criterion of metals. Springer Vienna, 1999.

\end{enumerate}
}
