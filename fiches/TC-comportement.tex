\subsection{Comportements non-linéaires des solides}

\paragraph{Professeur(s):} Djimedo Kondo, \url{http://www.dalembert.upmc.fr/home/kondo/}.\newline E-Mail: djimedo.kondo@sorbonne-universite.fr.
%Cet enseignement vise à donner aux étudiants un cadre unifié pour la formulation des lois de comportements mécaniques non linéaires, et à leur considération dans le con-texte du calcul de structures. Le point de vue adopté est pour l’essentiel macrosco-pique et phénoménologique, avec cependant des éléments de micromécanique des matériaux qui viennent justifier ou enrichir ce point de vue. 
%Après un rappel des différentes classes de lois de comportements thermomécaniques non linéaires, on présentera le cadre thermodynamique des processus irréversibles, l’objectif principal étant la formulation de lois de comportements régis par des méca-nismes dissipatifs identifiés. Le rôle clef des 2 principes de la thermodynamique, et notamment de l’inégalité de Claussius-Duhem qui en résulte, sera souligné. 
%Suivra un exposé du cadre des Matériaux Standards généralisés (MSG) pour lequel la formulation des lois constitutives repose sur :
%\begin{itemize}
%\item un choix éclairé des variables d’état incluant les variables internes correspondant aux types de comportements considérés
%\item  l’introduction d’un potentiel thermodynamique (en général l’énergie libre d’Helmholtz), fonction de l’ensemble des variables d’état et décrivant l’état thermo-dynamique du matériau
%\item  le choix d’un potentiel de dissipation fournissant les lois d’évolution des variables internes. 
%\end{itemize}

%Ce cadre des MSG servira ensuite à construire et étudier de manière unifiée les grandes classes de modèles de comportements thermomécaniques, allant de la ther-moélasticité à la thermo-elastoviscoplasticité. On évoquera brièvement la mécanique de l’endommagement. 
%Seront également décrites les équations régissant les évolutions thermo-mécaniques, nécessitées par la résolution de problèmes de structures qui jalonneront l’ensemble de l’enseignement. Ceci sera systématiquement complété par l’exposé du cadre variationnel associé.

\paragraph{Contenu de l’Unité d’Enseignement:}
\begin{itemize}
\item	Bref rappel des concepts de base de la mécanique des milieux continus, et des divers comportements des matériaux de structure
\item	Lois de bilan en thermomécanique des milieux continus : bilan de quantité de mouvement, bilan d’énergie et bilan d’entropie. Présentation de l’inégalité de Clausius-Duhem, et des dissipations (intrinsèque et thermique)
\item	Cadre des matériaux standards généralisés : méthode générale de formulation des lois de comportement; rôle de l’Inégalité de Clausius-Duhem. Notions de variables d’état, de variables internes, de fonctions d’état et de potentiels thermodynamiques. Equation de la chaleur.
Construction et identification des grandes classes de modèle de comportement : rappels des modèles rhéologiques fondamentaux à base de ressort, patin et amortis-seur. Cas des matériaux thermoplastiques, des matériaux thermo-viscoélastiques (mo-dèle de Kelvin-Voigt, modèle de Maxwell)
\item	Elastoplasticité et applications : Formulation de modèles élastoplastiques par-faits. Prise en compte de l’écrouissage (isotrope et/ou cinématique). Calculs thermo-mécaniques et résolution de problèmes simples de structures élastoplastiques.
\item	Quelques notions sur des lois couplant élasticité et endommagement isotrope
\item	Comportements Elastoviscoplastiques : Présentation de quelques modèles de com-portement dépendant du temps; effet régularisant de la viscosité. 
\end{itemize}



\paragraph{Références bibliographiques}:

\begin{enumerate}
\item 
H. Ziegler, An introduction to thermoechanics, North Holland, 1983
\item P. Germain, Q. S. Nguyen, P. Squat, Continuum Thermodynamics, J. Appl. Mech., ASME 50, 1010-1021, 1983.
\item J. Lemaître, J. L. Chaboche, Mechanics of Solids Materials, Cambridge University Press, 1990
\item G. Maugin, The thermomechanics of plasticity and fracture, Cambridge University Press 1992
\item Q. S. Nguyen, Stability and Nonlinear Solids Mechanics, Wiley, 2000 
\item M. Fremond, Non Smooth Thermomechanics, Springer Verlag, 2002,
\item J. Lubliner, Plasticity Theory. Dover Publications Inc., Mineola, New York, 2008.
\item H. Maitournam, Matériaux et Structures inélastiques, Editions de l’Ecole Polytechnique, 2016 
\end{enumerate}