\subsection{Mécanique de la rupture fragile}

\paragraph{Professeur:} 
Jean-Baptiste Leblond, \url{http://www.dalembert.upmc.fr/home/leblond},\\ E-Mail: jbl[arobas]lmm[dot]jussieu[dot]fr


\paragraph{Objectifs de l'Unité d'Enseignement:}
Exposer les bases de la théorie de la rupture fragile, telle qu’elle est couramment utilisée dans les laboratoires de recherche et l’industrie de pointe (nucléaire, aéronautique, …) pour prédire et maîtriser la fissuration des matériaux.


%\emph{English : Expound the bases of the theory of brittle fracture mechanics, such as currently used in research institutions and high-tech industries (aeronautical, nuclear…) in order to predict and master cracking of materials.}


\paragraph{Contenu de l’Unité d’Enseignement:}
Le cours inclut 2 chapitres sur les connaissances de base et un 3ème un peu plus spécialisé: : 
\begin{itemize}
 \item Théorie d’Irwin du KIc,
 \item Théorie énergétique de Griffith,
 \item Propagation de fissures en mode mixte.
\end{itemize}

\paragraph{Pré-requis:}

\begin{itemize}
\item Mécanique des solides niveau Master 1 (niveau d'exigence 100\%)
\item Bonnes bases de mathématiques pratiques (algèbre et analyse élémentaires, équations différentielles, fonctions d’une variable complexe, calculs). (niveau d'exigence 50\%).
\end{itemize}

\paragraph{Mots clés:}
Matériaux élastiques, fissuration, singularités de contraintes, critère de propagation, approche énergétique, mode mixte, branchement de fissures

\paragraph{Compétences développées dans l’unité:}
A l’issue de l’UE, l’étudiant(e) dispose des outils de base de la mécanique de la rupture fragile, exposés de manière exhaustive et détaillée, lui permettant soit de satisfaire aux exigences d’un bureau d’études en mécanique soit d’entreprendre une thèse dans le domaine. 