\subsection{Introduction à l’homogénéisation en mécanique des milieux continus }

\paragraph{Professeur:\newline} 
Karam Sab, \url{https://navier.enpc.fr/SAB-Karam}, E-Mail : 
karam.sab@enpc.fr



\paragraph{Objectifs de l'Unité d'Enseignement: }
Le comportement des matériaux peut être modélisé de deux manières complémentaires : la démarche phénoménologique et le changement d'échelle. L'approche phénoménologique consiste à identifier expérimentalement des lois de comportement à l'échelle d'un élément de volume représentatif du matériau, alors que les techniques de changement d'échelle se proposent de calculer des estimations du comportement du matériau à partir du comportement de ses constituants et de leurs fractions volumiques. L'objet de ce cours est d'introduire les concepts et techniques de base nécessaires pour effectuer, grâce au changement d’échelle, une homogénéisation d’un matériau hétérogène élastique linéaire.

\paragraph{Contenu de l’Unité d’Enseignement:}

\begin{itemize}
	\item Introduction des différentes échelles d'observation dans les solides hétérogènes. Notion de Volume Élémentaire Représentatif (VER).
	\item Conditions aux limites homogènes en déformation ou en contrainte. Tenseurs d'élasticité et de souplesse du VER.
	\item Bornes de Voigt et de Reuss. Cas du composite unidirectionnel.
	\item Méthodes approchées dans le cas de faibles concentrations d'inclusions. Aperçu des méthodes autocohérentes et du modèle de Mori-Tanaka.
	\item Cas des milieux à structure périodique.
\end{itemize}

\paragraph{Mots-clés:} 

Homogénéisation. Milieux périodiques. Micromécanique.



\paragraph{Pré-requis: }

Il est nécessaire de maîtriser la modélisation élastique linéaire des solides déformables

\paragraph{Compétences développées dans l’unité:}
Pratique de la modélisation en élasticité linéaire. Homogénéisation des matériaux élastiques.


\paragraph{Références bibliographiques:}
\begin{enumerate}
\item Hashin Z., Analysis of composite materials, a survey. J. Appl. Mech., 50, 481-505 (1983); 
\item Sanchez-Hubert J., Sanchez-Palencia E., Introduction aux méthodes asymptotiques et à l’homogénéisation, Masson, Paris, 1992; 
\item Sab K., On the homogenization and simulation of random materials. Eur. J. Mech. A/Solids, 11 (5), 585-607. 1992; 
\item Nemat-Nasser S., Hori M., Micromechanics: Overall Properties of Heterogeneous Materials, North-Holland, 1993; 
\item Kozlov S.M., Olenik O., Zhikov V., Homogenization of Differential Operators, Springer Verlag,1994; 
\item Sab K., Propriétés homogénéisées des matériaux hétérogènes élastiques: définition et bornes. Actes des journées “ changement d’échelle ”. 7 et 8 juin 2000. Nantes. LCPC. 2000;
\item 
Bornert M., Bretheau T., Gilormini P. (Eds), Homogénéisation en mécanique des matériaux, Hermes, Paris, 2001.
\end{enumerate}